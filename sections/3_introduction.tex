\section{Introduction}

% Include
% 1) What is the problem?
% 2) What do we know about low and varying use of thrombolysis
% 3) What do we not know
% 4) How are we addressing what we don't know

% 1) What is the problem?

Stroke remains one of the top three global causes of death and disability \cite{feigin_global_2021}. Despite reductions in age-standardised rates of stroke, ageing populations are driving an increase in the absolute number of strokes \cite{feigin_global_2021}. Across Europe, in 2017, stroke was found to cost healthcare systems \texteuro 27 billion, or 1.7\% of health expenditure \cite{luengo-fernandez_economic_2020}. Thrombolysis with recombinant tissue plasminogen activator, can significantly reduce disability after ischaemic stroke, so long as it is given in the first few hours after stroke onset \cite{emberson_effect_2014}. Despite thrombolysis being of proven benefit in ischaemic stroke, use of thrombolysis varies significantly both between and within European countries \cite{aguiar_de_sousa_access_2019}. In England and Wales the national stroke audit reported that in 2021/22, 20 years on from the original European Medicines Agency licencing of alteplase for acute ischaemic stroke, thrombolysis rates for emergency stroke admissions varied from just 1\% to 28\% between hospitals \cite{sentinel_national_stroke_audit_programme_ssnap_2022}, with a median rate of 10.4\% and an inter-quartile range of 8\%-13\%, against a 2019 NHS England long term plan that 20\% of patients of emergency stroke admissions should be receiving thrombolysis \cite{nhs_long_term_plan_2019}.

The NHS plan for improving stroke care also sets a target that patients should receive thrombolysis within 60 minutes of arrival, but ideally within 20 minutes (12). Whilst this speed of thrombolysis, called door-to-needle time, provides an ambitious target, it has been shown to be achievable as Helsinki University Central Hospital has reported a median door-to-needle time of 20 minutes, with 94\% of patients treated within 60 minutes \cite{meretoja_reducing_2012}.

% 2) What do we know about low and varying use of thrombolysis

Studies have shown that reasons for low and varying thrombolysis rates are multi-factorial. Reasons include late presentation \cite{aguiar_de_sousa_access_2019}, lack of expertise \cite{aguiar_de_sousa_access_2019} or lack of clear protocols or training \cite{carter-jones_stroke_2011}, delayed access to specialists \cite{kamal_delays_2017}, and poor triage by ambulance or emergency department staff \cite{carter-jones_stroke_2011}. For many factors, the establishment of primary stroke centres has been suggested to improve the emergency care of patients with stroke and reduce barriers to thrombolysis \cite{carter-jones_stroke_2011}, with a centralised model of primary stroke centres leading to increased likelihood of thrombolysis \cite{lahr_proportion_2012, morris_impact_2014, hunter_impact_2013}. 

In addition to organisational factors, clinicians can have varying attitudes to which patients are suitable candidates for thrombolysis. In a discrete choice experiment \cite{de_brun_factors_2018}, 138 clinicians considered hypothetical patient vignettes, and responded as to whether they would give the patients thrombolysis. The authors concluded that there was considerable heterogeneity among respondents in their thrombolysis decision-making. Areas of difference were around whether to give thrombolysis to mild strokes, to older patients beyond 3 hours from stroke onset, and when there was pre-existing disability.

Based on national audit data from three years of emergency stroke admissions, we have previously built models of the emergency stroke pathway using clinical pathway simulation to examine the potential scale of the effect of changing two aspects of the stroke pathway performance (1. the in-hospital process speeds, and 2. the proportion of patients with a determined stroke onset time), and using machine learning to examine the effect of replicating clinical decision-making around thrombolysis from higher thrombolysing hospitals to lower thrombolysing hospitals \cite{allen_using_2022, allen_use_2022}. The machine learning model learned whether any particular patient would receive thrombolysis in any particular emergency stroke centre. Using these models we found that it would be credible to target an increase in average thrombolysis in England and Wales, from 11\% to 18\%, but that each hospital should have its own target, reflecting differences in local populations. We found that the largest increase in thrombolysis use would come from replicating thrombolysis decision-making practice from higher to lower thrombolysing hospitals. Two other important factors influencing thrombolysis rates were determination of stroke onset time in some hospitals, and improving the speed of the in-hospital thrombolysis pathway.

% 3) What do we not know

In our previous work we established that we could predict the use of thrombolysis in patients arriving within 4 hours of known stroke onset with 84.3\% accuracy \cite{allen_use_2022}. We could then ask the question "What if this patient attended another hospital - would they likely be given thrombolysis?" In that work, when we altered selection of patients for thrombolysis, we were only predicting thrombolysis choice; we were not examining outcomes after stroke with and without thrombolysis, which left open the question of whether stroke teams with higher thrombolysis use were likely to be achieving better outcomes. In this paper we extend the pathway model to examine outcomes after applying the predicted decision-making from higher thrombolysing hospitals, including a health economics analysis. The work described here is companion work to studies on how the observed effect of thrombolysis compares with clinical trials \cite{pearn_thrombolysis_2024} and a study on whether those receiving thrombolysis are the the ones who will most likely benefit from it \cite{pearn_are_2024}.